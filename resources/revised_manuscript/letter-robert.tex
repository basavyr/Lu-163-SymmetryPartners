\documentclass[11pt,a4paper]{article}

\usepackage{amsmath}

\begin{document}

Thank you for reading our manuscript and the valuable remarks. Here are the added modifications along the line suggested by your report.

1) The signature splitting was discussed on the page 3, the paragraph ”The quantity ...”. Indeed, we have mentioned that the total splitting comes from summing two effects determined by the single particle and the core, respectively. The total splitting is consistent, indeed, with the estimate of Jensen (ref. [3])

2) First of all, your remark that negative parity states do not exist since there is no octupole deformation is not correct, to my understanding. Octupole deformation implies seven degrees of freedom while here the core has only three (Euler angles fixing the position of the intrinsic frame of reference with respect to the laboratory one), involved in the three angular momentum components. Here, the parity operator acts in the space of angular momenta. The fact that the triaxial rotor has negative parity states is proved in the paragraph ”Remarkable the fact...” on page 3, the second column. These arguments may be found in any textbook on the triaxial rigid rotor. The case treated microscopically by Nazarewicz et. al. was also commented. \textbf{Furthermore, the simplex quantum number (that is, the eigenvalue of the $S_1$ operator) from the quoted paper is not a relevant quantity here, since, within the present formalism, the deformation that characterizes the nuclear system is the quadrupole deformed mean-field (with its single-particle strength parameter $V$ which is proportional to the axial deformation parameter $\beta$), and no odd-multipole terms are included. In fact, an odd-multipole term in the deformed potential would break the $D_2$ symmetry of the triaxial rotor Hamiltonian. As a result, signature and parity remain good quantum numbers in the current model.}

3) In the beginning of your report you say that the accepted interpretation of the TSD bands in $^{163}$Lu is: TSD1 $i_{13/2}$, $\alpha = -1/2$ quasi-proton, 0 wobbling phonon, ..., etc. To our understanding, for this isotope the yrast band TSD1 is $i_{13/2}$,  $\alpha= +1/2$, ..., etc. Am I right? \textbf{It is known that the wobbling bands have an alternating signature behavior (that is yrast-favored, yrare-unfavored and so on). In the case of $^{163}$Lu nucleus the signature is therefore alternating between $\alpha_f=+1/2$ (favored) and $\alpha_u=-1/2$ (unfavored), provided that the odd-particle is $\pi(i_{13/2})$. See for example \emph{I. Hamamoto, Physical Review C, V65, 044305, (2002)}, \emph{D.R. Jensen et al., Nuclear Physics A 703 (2002)} or, more recently, \emph{S. Chakraborty et. al., Physics Letters B 811 135854 (2020)}. Indeed, our proposal is consistent with the formalism of the wobbling bands in odd mass nuclei (Table II from the manuscript shows an alternating character of the signature).}

4) The report says that our model is unphysical(!). This statement is based on the fact that it differs from the picture proposed earlier by other authors. \textbf{This contradicts our statement that the proposed formalism is realistic since describes the energies of the 62 states very accurately, better than any other known semi-classical approach. Moreover, the fitting procedure predicts a triaxial deformation parameter $\gamma=20^\circ$, which is fully consistent with previous calculations using the Ultimate Cranker approach. Then is the comparison with the data a criterion for appraising whether an approach is physical or not? I don’t think it is fair to say that a manuscript is not suitable for publication for the only reason that is not on the line of some previously published papers.}

\textbf{
5) Regarding the three-quasiparticle interpretation for TSD4 which arises from the investigation of \emph{Jensen et al., Eur. Phys. J. A 19, 173 (2004)}, indeed the argument that TSD4 has rotational properties that "are different" from TSD1-3 lead to the conclusion that TSD4 is not of wobbling type (the rotational properties of the bands were given in terms of alignment, the dynamic moment of inertia, and the excitation energy relative to a reference, i.e., Figs. 13-16). However, when they have discussed the alignment of the four bands (i.e., $i_x=I_x-I_\text{ref}$, where $I_\text{ref}=J_0\omega+J_1\omega^3$), it was mentioned that the behavior of TSD4 relative to the TSD1-3 group is very sensitive to the magnitude of the parameters $J_0$ and $J_1$, so the large difference in $i_x$ can be diminished. Moreover, regarding the dynamic MOI $\mathcal{J}^{(2)}$, the fourth band seems to have quite similar behavior when compared to the previous three bands (see Fig. 13 from \emph{D.R. Jensen et al., Nuclear Physics A 703 (2002)}, and the paragraph at page 18 starting with "The TSD bands are all characterized by very similar dynamic moments of inertia..."). Based on these qualitative investigations, we do not believe that it is completely excluded the fact that TSD4 could belong in the same family of rotational bands as TSD1-3, having thus a wobbling character.}

\end{document}