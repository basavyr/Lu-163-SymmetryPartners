\documentclass[%
 reprint,
 amsmath,
 amssymb,
 aps,
]{revtex4-2}

\usepackage{graphicx}% Include figure files
\usepackage{dcolumn}% Align table columns on decimal point
\usepackage{bm}% bold math
\usepackage{lipsum}
\usepackage{physics}
\usepackage{xcolor}


\bibliographystyle{apsrev4-2}


\begin{document}

%\preprint{APS/123-QED}
\title{Manuscript Title:\\Sub-Title of manuscript }% Force line breaks with \\

\author{author1}%
 \affiliation{Authors' institution and/or address}%
 \email{email@address.tbd}
\author{author1}%
\affiliation{%
Authors' institution and/or address}%


\date{\today}

\begin{abstract}
The wobbling spectrum of $^{163}$Lu is described through a novel approach, starting from a triaxial rotor model within a semi-classical picture, and obtaining a new set of equations for all four rotational bands that have wobbling character. Redefining the band structure in the present model is done by adopting the concepts of Signature Partner Bands and Parity Partner Bands. Indeed, describing a wobbling spectrum in a even-odd nucleus through signature and parity quantum numbers is a unique interpretation of the triaxial super-deformed structures in the chart of nuclides.
\end{abstract}

\maketitle


\section{Introduction}

Triaxiality in nuclear systems represented a great challenge over the last decades for the Nuclear Physics community due to its elusive character, however, a tremendous progress has been made in the recent years, both theoretically and experimentally. Regarding its fingerprints, it is a widely known and accepted fact that the phenomenon of \emph{Wobbling Motion} (WM) is a clear signature for triaxial shapes across the chart of nuclides. Although it was firstly predicted theoretically for even-even nuclei \cite{bohr1998nuclear}, this collective mode has been discovered and confirmed experimentally in several even-odd nuclei, with $^{163}$Lu being considered the best \emph{wobbler}, mainly due to its relatively rich spectrum in terms of wobbling bands (with four triaxial super-deformed bands TSD1,2,3, and 4, with TSD1 as the ground state - yrast - band and three wobbling excitations). Indeed, it was shown \cite{odegaard2001evidence}, \cite{jensen2002wobbling} that stable triaxial shapes occur in this isotope due to the aligned $i_{13/2}$ proton which drives the nucleus up to very large deformation. With time, several neighboring nuclei were identified as wobblers (i.e., $^{161,165,167}$Lu, and recently the nuclei $^{135}$Pr and $^{187}$Au {\color{red}CITS REQ}).

Wobbling motion (WM) is defined as a precession motion of the total angular momentum of a triaxial rigid body amended with a oscillation (around a fixed position) of its projection onto the quantization axis. This type of motion has a collective character and it is caused by the asymmetry in the magnitudes of the moments of inertia around the principal axes for a triaxial rotor. At a given spin, a uniform rotation around the axis with the largest moment of inertia will correspond to a minimal energy. With a slight increase in energy, this axis starts to process about the space-fixed angular momentum axis. WM appears as a sequence of rotational bands with $\Delta I=2$, where each band emerges from successive excitations of wobbling phonons $n_w$. Two adjacent wobbling bands $n_w$ and $n_{w+1}$ are characterized by strong $\Delta I=1$ transitions with E2 component.

In the case of odd-odd nuclei, it is the odd nucleon which couples to an even-even triaxial core, driving the entire nuclear system to very large deformations $(\beta_2,\gamma$) attaining a nuclear stability. The nature if the quasiparticle plays an important role in the alignment of the nucleon and the triaxial even-even core. If the nucleon has a particle nature, which emerges from the bottom of a deformed $j$-shell, then its angular momentum aligns with the short $s$-axis of the rotor, since this maximizes the overlap of its density distribution with the triaxial core, making the total energy minimum. On the other hand, for a hole emerging from the top of a deformed $j$-shell, the particle's angular momentum will align with the long $l$ axis of the triaxial rotor, since this will minimize the overlap of its density distribution with the even-even core. Moreover, if the quasiparticle emerges from the middle of a $j$-shell, then it will align its angular momentum with the intermediate $m$ axis of the triaxial core. One can see that in all cases, the alignment will minimize the total energy, which in fact will make the entire achieve a nuclear stability.

Depending on the odd quasiparticle motion inside the deformed mean field generated by the core, there are two types of WM which might occur; namely a motion with \emph{longitudinal} character and a motion with \emph{transverse} character \cite{frauendorf2014transverse}. If the nucleon aligns its angular momentum with the $m$ axis, then the nucleus has a longitudinal wobbling type of motion, otherwise, if the odd nucleon aligns its angular momentum perpendicular to the $m$ axis (i.e., along the $s$ or $l$ axes). It is considered that almost all the even-odd wobbler have a transverse character {\color{red}CITS REQ}, including the wobbling spectrum of $^{163}$Lu. Reasoning behind that is enforced by the argument that the wobbling energy has a decreasing behavior with respect to the increase in total spin $I$. However, there is an ongoing discussion in the literature on whether these nuclei have a defined wobbling character and depending on the models used to describe the nuclear system, some aspects are still unclear {\color{red}CITS REQ}.

Within a previous model \cite{raduta2020towards}, a description of the wobbling phenomenon in $^{163}$Lu was successfully made, describing the wobbling energies and the transition probabilities for all four triaxial super-deformed (TSD) bands, namely TSD1, TSD2, TSD3, and TSD4. Therein, the calculations were based on a particle-triaxial rotor that was solved on a semi-classical formalism. The band structure was described in terms of two ground states given by the coupling of the core with an odd $j=i_{13/2}$ proton, one excited wobbling band $n_w=1$ from the coupling of the core with the same proton, and one ground state band obtained from the coupling of the core and a different valence nucleon, namely the $j=h_{9/2}$ nucleon. More precisely, the TSD1 and TSD2 are considered as ground states (so zero phonon excitations) for the nucleus, with TSD1 being a sequence of spins $\mathbf{I}=\mathbf{R}+\mathbf{j}$ with the core's angular momentum as an even integer sequence $\mathbf{R}=0,2,4,\dots$ and TSD2 is a spin sequence generated from a core-particle-coupling with odd integer spin of the core $\mathbf{R}=1,3,5,\dots$.

\section{Theoretical Framework}

\bibliography{biblio}% Produces the bibliography via BibTeX.

\end{document}