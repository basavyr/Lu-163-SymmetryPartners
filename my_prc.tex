\documentclass[%
 reprint,
 amsmath,
 amssymb,
 aps,
]{revtex4-2}


\usepackage{graphicx}% Include figure files
\usepackage{dcolumn}% Align table columns on decimal point
\usepackage{bm}% bold math
\usepackage{lipsum}
\usepackage{physics}
\usepackage{xcolor}


\bibliographystyle{apsrev4-2}

\begin{document}

\title{Parity Partner Bands in $^{163}$Lu: \\ A novel approach in describing the negative parity states from its wobbling spectrum}% Force line breaks with \\

\author{Robert Poenaru}%
 \email{robert.poenaru@drd.unibuc.ro}
 \affiliation{Doctoral School of Physics, University of Bucharest}%
 \affiliation{Horia Hulubei National Institute of Nuclear Physics and Engineering, Magurele}%
\author{Apolodor Aristotel Raduta}%
\affiliation{Horia Hulubei National Institute of Nuclear Physics and Engineering, Magurele}%
\affiliation{Academy of Romanian Scientists, Bucharest}%

\date{\today}

\begin{abstract}
The wobbling spectrum of $^{163}$Lu is described through a novel approach, starting from a triaxial rotor model within a semi-classical picture, and obtaining a new set of equations for all four rotational bands that have wobbling character. Redefining the band structure in the present model is done by adopting the concepts of Signature Partner Bands and Parity Partner Bands. Indeed, describing a wobbling spectrum in a even-odd nucleus through signature and parity quantum numbers is a unique interpretation of the triaxial super-deformed structures in the chart of nuclides.
\end{abstract}

\maketitle


\section{Introduction}

Triaxiality in nuclear systems represented a great challenge over the last decades for the Nuclear Physics community due to its elusive character, however, a tremendous progress has been made in the recent years, both theoretically and experimentally. Regarding its fingerprints, it is a widely known and accepted fact that the phenomenon of \emph{Wobbling Motion} (WM) is a clear signature for triaxial shapes across the chart of nuclides. Although it was firstly predicted theoretically for even-even nuclei \cite{bohr1998nuclear}, this collective mode has been discovered and confirmed experimentally in several even-odd nuclei, with $^{163}$Lu being considered the best \emph{wobbler}, mainly due to its relatively rich spectrum in terms of wobbling bands (with four triaxial super-deformed bands TSD1,2,3, and 4, with TSD1 as the ground state - yrast - band and three wobbling excitations). Indeed, it was shown \cite{odegaard2001evidence}, \cite{jensen2002wobbling} that stable triaxial shapes occur in this isotope due to the aligned $i_{13/2}$ proton which drives the nucleus up to very large deformation. With time, several neighboring nuclei were identified as wobblers (i.e., $^{161,165,167}$Lu, and recently the nuclei $^{135}$Pr and $^{187}$Au {\color{red}CITS REQ}).

Wobbling motion (WM) is defined as a precession motion of the total angular momentum of a triaxial rigid body amended with a oscillation (around a fixed position) of its projection onto the quantization axis. This type of motion has a collective character and it is caused by the asymmetry in the magnitudes of the moments of inertia around the principal axes for a triaxial rotor. At a given spin, a uniform rotation around the axis with the largest moment of inertia will correspond to a minimal energy. With a slight increase in energy, this axis starts to process about the space-fixed angular momentum axis. WM appears as a sequence of rotational bands with $\Delta I=2$, where each band emerges from successive excitations of wobbling phonons $n_w$. Two adjacent wobbling bands $n_w$ and $n_{w+1}$ are characterized by strong $\Delta I=1$ transitions with E2 component.

In the case of odd-odd nuclei, it is the odd nucleon which couples to an even-even triaxial core, driving the entire nuclear system to very large deformations $(\beta_2,\gamma$) attaining a nuclear stability. The nature if the quasiparticle plays an important role in the alignment of the nucleon and the triaxial even-even core. If the nucleon has a particle nature, which emerges from the bottom of a deformed $j$-shell, then its angular momentum aligns with the short $s$-axis of the rotor, since this maximizes the overlap of its density distribution with the triaxial core, making the total energy minimum. On the other hand, for a hole emerging from the top of a deformed $j$-shell, the particle's angular momentum will align with the long $l$ axis of the triaxial rotor, since this will minimize the overlap of its density distribution with the even-even core. Moreover, if the quasiparticle emerges from the middle of a $j$-shell, then it will align its angular momentum with the intermediate $m$ axis of the triaxial core. One can see that in all cases, the alignment will minimize the total energy, which in fact will make the entire achieve a nuclear stability.

Depending on the odd quasiparticle motion inside the deformed mean field generated by the core, there are two types of WM which might occur; namely a motion with \emph{longitudinal} character and a motion with \emph{transverse} character \cite{frauendorf2014transverse}. If the nucleon aligns its angular momentum with the $m$ axis, then the nucleus has a longitudinal wobbling type of motion, otherwise, if the odd nucleon aligns its angular momentum perpendicular to the $m$ axis (i.e., along the $s$ or $l$ axes). It is considered that almost all the even-odd wobbler have a transverse character {\color{red}CITS REQ}, including the wobbling spectrum of $^{163}$Lu. Reasoning behind that is enforced by the argument that the wobbling energy has a decreasing behavior with respect to the increase in total spin $I$. However, there is an ongoing discussion in the literature on whether these nuclei have a defined wobbling character and depending on the models used to describe the nuclear system, some aspects are still unclear {\color{red}CITS REQ}.

Within a previous model \cite{raduta2020towards}, a description of the wobbling phenomenon in $^{163}$Lu was successfully made, describing the wobbling energies and the transition probabilities for all four triaxial super-deformed (TSD) bands, namely TSD1, TSD2, TSD3, and TSD4. Therein, the calculations were based on a particle-triaxial rotor that was solved on a semi-classical formalism. The band structure was described in terms of two ground states given by the coupling of the core with an odd $j=i_{13/2}$ proton, one excited wobbling band $n_w=1$ from the coupling of the core with the same proton, and one ground state band obtained from the coupling of the core and a different valence nucleon, namely the $j=h_{9/2}$ nucleon. More precisely, the TSD1 and TSD2 are considered as ground states (with zero wobbling phonon excitations) for the nucleus, with TSD1 made up of a spins sequence $\mathbf{I}=\mathbf{R}+\mathbf{j}$ with the core's angular momentum as an even integer sequence $\mathbf{R}=0,2,4,\dots$ and TSD2 is a spin sequence generated from a core-particle-coupling with odd integer spin of the core $\mathbf{R}=1,3,5,\dots$. TSD3 has indeed a wobbling phonon character, and it is obtained by exciting the states belonging to TSD2 by a specific one phonon operator. As mentioned, TSD4 consists in a collection of energies that emerge from the core's coupling with a a valence nucleon from a different $j$-shell (i.e., $j=h_{9/2}$). In terms of the TSD4 ground state band, the spin sequence $\mathbf{I}=\mathbf{R}+\mathbf{j}$ has odd integers for the core's angular momentum $\mathbf{R}=1,3,5,\dots$, just like the TSD2.

It is worth mentioning some key points from \cite{raduta2020towards} and \cite{raduta2020new} that would be relevant as a starting point for the present study. i) Indeed, in the mentioned work it was shown that TSD1 and TSD2 are signature partner bands \cite{sun1994varied}, with the yrast band being characterized by positive signature number $\alpha=1/2$ (favored partner), while the second band has negative signature $\alpha=-1/2$ (unfavored partner). The signature quantum number is specific to deformed nuclei, and moreover it indicates the invariance of a nuclear system with quadrupole deformation under rotations with $\pi$ around the principal axes (with the observation that if the nucleus has axial symmetry, then the only rotations around the principal axes other than the symmetry one can have a good signature quantum number). Potential energy calculations that were performed for the case even-odd triaxial nuclei within \cite{raduta2016specific}-\cite{raduta2017semiclassical}, show that the minima are deep, so that states from TSD2 cannot tunnel to a secondary minimum, as a result, both bands share similar properties. ii) The fourth band in $^{163}$Lu was reconsidered as a zero-phonon wobbling excitation, since it belongs to a different coupling scheme and that also reflected in the calculations regarding the excitation energies of its states. More precisely, a different sets of moments of inertia, triaxiality parameter, and single-particle potential strength had to be considered in the numerical calculations, since the triaxial-core-coupling differed from that of TSD1,2 and 3 (see Eq. (12) in \cite{raduta2020new}). iii) Calculations regarding the stationary points of the energy function associated to the triaxial even-odd nucleus lead to the possibility to construct a phase diagram, which could show regions where both longitudinal and transverse wobbling character might occur, but also regions in that space were wobbling motion is forbidden. In fact, within those calculations, the model was able to predict that transverse wobbling is possible, but in an \emph{intermediate} picture, where the longitudinal character emerges as a final collective mode, from a state where the deformed system rotates around the $m$-axis. However, this is a limitation of the used Hamiltonian, and Appendix B from \cite{raduta2020new} shows that for spin-dependent moments of inertia and Coriolis-like interaction it is possible to consistently reproduce a decreasing wobbling energy as a function of rotational frequency or spin, specific to a transverse-like character of the WM. iv) The used formalism applied to the study of $A=163$ Lu isotope is not based on microscopic calculations which could justify the fact that fourth band has negative parity $\pi_4=-1$. As a matter of fact, the parity of the odd system is given by the product of parities associated to the core and the single particle (i.e., positive parity attributed to the core and negative parity that is attributed to the valance nucleon). From a microscopic point of view, the particle-hole excitations involve one positive single parity state that belongs to the core and one negative parity that belongs to the odd-particle.

Starting from the model sketched above, some further investigations on the actual band structure of $^{163}$Lu had been developed, with a focus on the energy spectrum, and whether it is possible to give a proper interpretation of the parity of all four bands. Namely, a great interest was given to TSD4, in order to see if its coupling might be strongly related to the first three bands. As such, an attempt on describing the excited states from TSD4 by adopting the same $\mathbf{I}=\mathbf{R}+\mathbf{j}$ coupling scheme as for TSD2 (where the valence nucleon from the $i_{13/2}$ orbital is coupled to the even-even triaxial core) is made in this paper. Results show that TSD2 and TSD4 are \emph{parity partner bands}, where they both have the same core spin sequence $\mathbf{R}=1,3,5,\dots$ but opposite parity numbers; $\pi_4=-\pi_2$. The concept of (negative) parity for even-even and even-odd deformed nuclei has been extensively studied over the last years (See Refs. \cite{raduta2006description}, \cite{raduta2006positive}, \cite{raduta2006simultaneous}, and \cite{bizzeti2008description}). 

Parity number $\pi$ describes the symmetry of the underlying Hamiltonian under a space reflection operation. Moreover, it is suggested that negative parity band structures in nuclei might be related to the existence of octupole deformations \cite{chasman1980incipient},\cite{asaro1953complex}. However, no concrete evidence that the states belonging to TSD4 could lead to pear shaped nuclear surfaces in $^{163}$Lu has been found so far. As a result, the present work asserts the possibility that indeed TSD4 belongs to the same family of wobbling bands as the previous three (enforced by the calculations that show similar behavior of the dynamic moment of inertia, alignment and the electromagnetic transitions performed by the same team in \cite{raduta2020towards}), while the pair (TSD2,TSD4) emerge as a collection of states with similar spins but opposite parity. This implies that due to a similar coupling scheme between $\mathbf{R}$ and $\mathbf{j}$ across all four bands (as opposed to previous models that considered the triaxial core+odd $j=h_{9/2}$ coupling) , a similar set of moments of inertia can be attributed within the calculations regarding the energies of the states, and moreover, the nucleus has the same triaxiality parameter $\gamma$, while the valence nucleon (that is the odd proton from $i_{13/2}$ orbital) is moving in a quadrupole deformed mean field generated by the core, with the same strength parameter $V$ \cite{shou2009coupling}. In fact, the present model considers these quantities as free parameters and they represent an unique set associated with all four bands. Numerical calculations will determine the values for the MOIs $\mathcal{I}_1$, $\mathcal{I}_2$, $\mathcal{I}_3$, $\gamma$ and $V$ (four free parameters) in such a way that the energy spectrum obtained will consistently reproduce the experimental data.

Regarding the nature of the bands TSD1 and TSD3, in the present model they remain unchanged. More precisely, the yrast band with zero-phonon excitation TSD1 that emerges as a collection of states with even-numbers for the core's spin sequence, and the one-wobbling-phonon excitation band TSD3 with a state $I$ which is built on top of TSD2 by acting on a $(I-1)\in \text{TSD2}$ state with a phonon operator.

Further investigations using this novel approach were devoted to the study of nuclear motion for even-odd triaxial systems. Working within a semi-classical picture allows one to keep a close contact with the classical features of the nuclear system dynamics, and to have a consistent description with the motion of a nucleus and that of a classical rigid rotor. Several studies \cite{frauendorf2014transverse}, \cite{lawrie2020tilted} showed the \emph{trajectories} of a simple rotor for a given total angular momentum and inertia parameters of the triaxial ellipsoid. One can view these trajectories as the allowed states of existence which the isotope might have with the increase in angular momentum. They are generated by the motion of the angular momentum vector, which from a classical point of view, it is the intersection of the angular momentum sphere and that of the energy ellipsoid. From the conservation of these two quantities, it is possible to extract information with regards to the intersection lines within the angular momentum space, and even predict certain trajectories with larger rotational spin and energy.
 
The present paper is structured as follows: i) an introduction on the importance of wobbling motion together with some recent results that the team achieved, and the new attempt on describing the wobbling spectrum of $^{163}$Lu by the extension of previous work with parity partners concept was presented in this section. ii) Following up, a brief description of the theoretical framework used within the calculations of the energy spectrum and trajectories is given. It is shown that starting from a Particle Rotor Model based Hamiltonian one can obtain semi-classical formulas for all four bands in the isotope. The energy function of the deformed system is written in terms of the spherical angles $(\theta,\varphi)$, and a self-consistent expression that will be used for the determination of the classical trajectories will be obtained. From the parametrization of the energies, a set of moments of inertia are obtained, which will be used for the graphical representation of the mentioned trajectories. iii) Numerical results with the free parameters and plots with the four triaxial bands as function of total spin $I$ will be represented in the section dedicated to the results. The energy function will be pictorially shown in terms of its critical points, with the focus on finding its minima. As a final step throughout the numerical analysis performed on the $^{163}$Lu isotope using this new model, the classical trajectories within the 3-dimensional space generated by the total angular momentum will be graphically represented, together with their projections onto each of the three planes. iv) Several concluding remarks will be drawn at the end of the paper.

\section{Theoretical Framework}

The Hamiltonian of $^{163}$Lu has a particle-rotor character, namely it describes the interaction between an even-even triaxial core and a single nucleon that moves in the quadrupole deformed mean field generated by the core.

\begin{align}
    H=H_\text{Rot}+H_\text{sp}\ . \label{hamiltonian_formula}
\end{align}

The first term represents the triaxial rotor Hamiltonian that is associated with the core angular momentum $\mathbf{R}=\mathbf{I}-\mathbf{j}$, and it is defined by the inertial parameters $A_k$.

\begin{align}
    H_\text{Rot}=\sum_{i=1,2,3}A_i\left(I_i-j_i\right)^2,
\end{align}
 
 where the inertial parameters $A_i$ are related to the moments of inertia through the formula $A_i=\frac{1}{2\mathcal{I}_i}$. The moments of inertia are associated to the principal axes of the triaxial ellipsoid, namely the $i$-th MOI $\mathcal{I}_i$ corresponds to the $i$-axis of the system.

The single-particle term from Eq. \ref{hamiltonian_formula} is defined in terms of the triaxiality parameter $\gamma$, which defines the ratios between MOIs and it is a measure of asymmetry, and the potential strength $V$, that is related to the deformation of quadrupole type (in \cite{shou2009coupling}, a determination of the interaction parameter $V$ is determined in terms of the quadrupole deformation $\beta_2$ using the single-$j$ shell model).

\begin{align}
    H_\text{sp} = \frac{V}{j(j+1}\left[\cos\gamma\left(3j_3^2-\mathbf{j}^2\right)-\sqrt{3}\sin\gamma\left(j_1^2-j_2^2\right)\right]+\epsilon_j. \label{sp_hami}
\end{align}

The term $\epsilon_j$ from Eq. \ref{sp_hami} represents the single particle energy associated with the orbital from which the valence nucleon belongs to.

With the above Hamiltonian, and by following the recipe described in section II from \cite{raduta2020new}, it is possible to obtain a set of classical equations of motion for $\mathcal{H}$ - the classical energy, defined as the expectation value of $H$ with the trial function used throughout the calculations (See Eqs. (1)-(9) from \cite{raduta2020new}). The energy function $\mathcal{H}$ is minimal in the point $(\varphi,r;\psi,t)=(0,I;0,j)\equiv P_0$ when the maximal moment of inertia for the triaxial nucleus is along the 1-axis. A harmonic motion of the system is obtained if one linearizes the equations of motion around $P_0$, with the frequency of the motion given by a dispersion equation, which under the restrictions $\mathcal{I}_1>(\mathcal{I}_2,\mathcal{I_3})$ admits two real and positive solutions. These solutions represent the \emph{wobbling frequencies} $\Omega_1$ and $\Omega_2$ that will be used for the analytic expressions of the excited spectrum of $^{163}$Lu. The notations used in the previous model differ from the current one, since there were two cases for consideration, namely the coupling of the core with $j_1=i_{13/2}$ and with $j_2=h_{9/2}$. Here, for simplicity, differentiation between the two frequencies will be done the indexes 1,2, but both solutions are associated to the unique $\mathbf{R}+\mathbf{j}$ coupling for all four TSD bands where $j=j_1$. In \cite{raduta2017semiclassical} it was shown that the solutions of the dispersion equation for $\Omega$ correspond to the motion of the core ($\Omega_1$) and the odd-particle ($\Omega_2$), and moreover, each wobbling quanta has a corresponding wobbling phonon number $n_{w_1}$ and $n_{w_2}$, respectively. Having thus the minimal energy $\mathcal{H}_min$ (given as the classical energy function evaluated at $P_0$ for a given total angular momentum), and the two wobbling frequencies (given as solutions to the dispersion equation associated to the harmonic motion of the system) with their corresponding wobbling phonon numbers, the energy spectrum of the studied isotope can be defined as:

\begin{align}
    E_I^\text{TSD1}=\epsilon_{j_1} + \mathcal{H}_\text{min}^{(I,j_1)}+\mathcal{F}_{00}^I \nonumber\ ,\\
    E_I^\text{TSD2}=\epsilon_{j_1} + \mathcal{H}_\text{min}^{(I,j_1)}+\mathcal{F}_{00}^I \nonumber\ ,\\
    E_I^\text{TSD3}=\epsilon_{j_1} + \mathcal{H}_\text{min}^{(I,j_1)}+\mathcal{F}_{10}^I \nonumber\ ,\\
    E_I^\text{TSD4}=\epsilon_{j_1} + \mathcal{H}_\text{min}^{(I,j_1)}+\mathcal{F}_{00}^I \label{wobbling_energies}
\end{align}

where the terms $\mathcal{F}_{ij}$ represent the wobbling frequencies of phononic character, associated to each band. These terms are related to the total angular momentum through the following expression:

\begin{align}
    F_{n_{w_1}n_{w_1}}=\frac{1}{2}\left(\Omega_1^I+\Omega_2^I\right) \label{phonons}
\end{align}

with the observation that for TSD2 and TSD4, due to their interpretation as being parity partners, a shift in the overall energy will appear (denoted hereafter with $\epsilon_2$ for TSD2 and $\epsilon_4$ for TSD4). These two quantities will be adjusted throughout the numerical calculations such that the energy spectrum is best reproduced. From the expression of Eq. \ref{phonons}, one can see what wobbling phonon numbers are associated to the frequencies, for each of the triaxial bands. These values are listed in the Table \ref{tabular_phonon_numbers}, where values of the parity and signatures are also shown.

\begin{table}
    \centering
  \begin{tabular}{lllll}
  \hline
Band & $n_{w_1}$ & $n_{w_2}$ & $\pi$ & $\alpha$ \\
\hline
\hline
TSD1 &     0      &       0    &     +1  &    +1/2      \\
TSD2 &    0       &       0    &   +1    &        -1/2  \\
TSD3 &     1      &     0      &    +1   &        +1/2  \\
TSD4 &     0      &     0      &    -1   &     +1/2    \\
\hline
\end{tabular}
    \caption{The wobbling phonon numbers, parities and signatures assigned for the triaxial bands in $^{163}$ within the model.}
    \label{tabular_phonon_numbers}
\end{table}


\bibliography{biblio}% Produces the bibliography via BibTeX.

\end{document}