\documentclass[%
 reprint,
 amsmath,
 amssymb,
 aps,
]{revtex4-2}

\usepackage{graphicx}% Include figure files
\usepackage{dcolumn}% Align table columns on decimal point
\usepackage{bm}% bold math
\usepackage{lipsum}
\usepackage{physics}



\bibliographystyle{apsrev4-2}


\begin{document}

%\preprint{APS/123-QED}
\title{Manuscript Title:\\Sub-Title of manuscript }% Force line breaks with \\

\author{author1}%
 \affiliation{Authors' institution and/or address}%
 \email{email@address.tbd}
\author{author1}%
\affiliation{%
Authors' institution and/or address}%


\date{\today}

\begin{abstract}
Abstract
\end{abstract}

\maketitle


\section{Introduction}

Triaxiality in nuclear systems represented a great challenge over the last decades for the Nuclear Physics community due to its elusive character, however, a tremendous progress has been made in the recent years, both theoretically and experimentally. Regarding its fingerprints, it is a widely known and accepted fact that the phenomenon of \emph{Wobbling Motion} (WM) is a clear signature for triaxial shapes across the chart of nuclides. Although it was firstly predicted theoretically for even-even nuclei \cite{bohr1998nuclear}, this collective mode has been discovered and confirmed experimentally in several even-odd nuclei, with $^{163}$Lu being considered the best \emph{wobbler}, mainly due to its relatively rich spectrum in terms of wobbling bands (with four triaxial super-deformed bands TSD1,2,3, and 4, with TSD1 as the ground state - yrast - band and three wobbling excitations). 

\section{Theoretical Framework}

\bibliography{biblio}% Produces the bibliography via BibTeX.

\end{document}