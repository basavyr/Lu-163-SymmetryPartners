

Wobbling motion (WM) is defined as a precession motion of the total angular momentum of a triaxial rigid body amended with a oscillation (around a fixed position) of its projection onto the quantization axis. This type of motion has a collective character and it is caused by the asymmetry in the magnitudes of the moments of inertia around the principal axes for a triaxial rotor. At a given spin, a uniform rotation around the axis with the largest moment of inertia will correspond to a minimal energy. With a slight increase in energy, this axis starts to process about the space-fixed angular momentum axis. WM appears as a sequence of rotational bands with $\Delta I=2$, where each band emerges from successive excitations of wobbling phonons $n_w$. Two adjacent wobbling bands $n_w$ and $n_{w+1}$ are characterized by strong $\Delta I=1$ transitions with E2 component.

In the case of odd-odd nuclei, it is the odd nucleon which couples to an even-even triaxial core, driving the entire nuclear system to very large deformations $(\beta_2,\gamma$) attaining a nuclear stability. The nature if the quasiparticle plays an important role in the alignment of the nucleon and the triaxial even-even core. If the nucleon has a particle nature, which emerges from the bottom of a deformed $j$-shell, then its angular momentum aligns with the short $s$-axis of the rotor, since this maximizes the overlap of its density distribution with the triaxial core, making the total energy minimum. On the other hand, for a hole emerging from the top of a deformed $j$-shell, the particle's angular momentum will align with the long $l$ axis of the triaxial rotor, since this will minimize the overlap of its density distribution with the even-even core. Moreover, if the quasiparticle emerges from the middle of a $j$-shell, then it will align its angular momentum with the intermediate $m$ axis of the triaxial core. One can see that in all cases, the alignment will minimize the total energy, which in fact will make the entire achieve a nuclear stability.

Depending on the odd quasiparticle motion inside the deformed mean field generated by the core, there are two types of WM which might occur; namely a motion with \emph{longitudinal} character and a motion with \emph{transverse} character \cite{frauendorf2014transverse,chen2019transverse}. If the nucleon aligns its angular momentum with the $m$ axis, then the nucleus has a longitudinal wobbling type of motion, otherwise, if the odd nucleon aligns its angular momentum perpendicular to the $m$ axis (i.e., along the $s$ or $l$ axes). It is considered that almost all the even-odd wobbler have a transverse character \cite{frauendorf2014transverse}, including the wobbling spectrum of $^{163}$Lu. Reasoning behind that is enforced by the argument that the wobbling energy has a decreasing behavior with respect to the increase in total spin $I$. However, there is an ongoing discussion in the literature on whether these nuclei have a defined wobbling character and depending on the models used to describe the nuclear system, some aspects are still unclear \cite{tanabe2017stability,frauendorf2018comment,tanabe2018reply}.

The present paper is structured as follows: i) an introduction on the importance of wobbling motion together with some recent results that the team achieved, and the new attempt on describing the wobbling spectrum of $^{163}$Lu by the extension of previous work with parity partners concept was presented in this section. ii) Following up, a brief description of the theoretical framework used within the calculations of the energy spectrum and trajectories is given. It is shown that starting from a Particle Rotor Model based Hamiltonian one can obtain semi-classical formulas for all four bands in the isotope. The energy function of the deformed system is written in terms of the spherical angles $(\theta,\varphi)$, and a self-consistent expression that will be used for the determination of the classical trajectories will be obtained. From the parametrization of the energies, a set of moments of inertia are obtained, which will be used for the graphical representation of the mentioned trajectories. iii) Numerical results with the free parameters and plots with the four triaxial bands as function of total spin $I$ will be represented in the section dedicated to the results. The energy function will be pictorially shown in terms of its critical points, with the focus on finding its minima. As a final step throughout the numerical analysis performed on the $^{163}$Lu isotope using this new model, the classical trajectories within the 3-dimensional space generated by the total angular momentum will be graphically represented, together with their projections onto each of the three planes. iv) Several concluding remarks will be drawn at the end of the paper.
